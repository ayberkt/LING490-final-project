\documentclass{article}
\usepackage{natbib}
\usepackage{hyperref}

\newcommand{\hmmURL}{https://hackage.haskell.org/package/hmm-0.2.1.1/docs/Data-HMM.html}

\title{\bf An HMM-based part-of-speech tagger for Turkish}
\author{Ayberk Tosun\\\texttt{tosun2@illinois.edu}}
\date{}

\begin{document}
\maketitle

\section{Motivation}
\label{sec:motivation}
My main motivation for undertaking this project is to understand the
principles underlying POS-tagging. Due to time and resource constraints, it is
not possible for me to build a state-of-the-art POS-tagger for
Turkish. Nevertheless, I expect my POS-tagger
to work with a decent accuracy that is adequate to be a starting point for
further development. Moreover, \citet{Korkut2015} began the development of a
Turkish NLP framework named ``Guguk,'' implemented in the programming language Haskell. I intend
to merge this project to Guguk, hence contributing towards
the existence of a Turkish NLP-framework in Haskell.

\section{Related Work}
\label{sec:related_work}

% Briefly survey the most salient prior work that relates to the problem you wish
% to solve. This section should cite relevant sources. This section should be at
% least one to two paragraphs in length.
The most important advancement in the computational linguistics of Turkish is
probably the NLP framework ``Zemberek'' by \citet{akin2007zemberek}, which has been
immensely useful for both the theory and practice of Turkish NLP. Although it started
out as an NLP tool for the Turkish language, Zemberek moved on to become a
a tool for processing all Turkic languages.

The widely cited \citet{oflazer1994tagging} outlines the implementation of a
tool for both morpheme glossing and POS tagging, that has ``98-99 \%'' accuracy. This
tool is based on a morphological specification of Turkish and makes use of a
morphological processor called ``PC-Kimmo'' \citep{antworth1991pc}.

In general, most of the literature relating to the computational processing of Turkish
pertain primarily to morphology---compared to the complexity of Turkish
morphology, its syntax seems to be a less interesting problem.
\section{Proposal}

I will implement a part-of-speech tagger for the Turkish language. As
mentioned in section~\ref{sec:related_work}, I will mostly focus on parts-of-speech
of words\footnote{I base my definition of a ``word'' on the orthographic space in
  most of the cases. There are certain exceptions to this, such as
  reduplication which I denote with the POS tag ``Dup''} and will not be accounting for morphology, unlike
the case with most of the serious research on Turkish NLP. There definitely are
problems with this approach---but fixing these would go beyond the scope of
this class

For programming the POS tagger I will be using Haskell, a statically-typed and purely
functional programming language, along with the
\href{\hmmURL}{\texttt{Data.HMM}} package.
The data, that is probably the most important component of this
project, comes from the METU-Sabanc{\i} treebank built by
\citet{oflazer2003building}.  I am planning to strip the POS tag for each word from
the treebank and use those to train the HMM

\bibliographystyle{apalike}
\bibliography{your_bibliography}

\end{document}
