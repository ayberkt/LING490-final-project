\documentclass{article}
\usepackage{natbib}
\usepackage{hyperref}

\title{\bf An HMM-based part-of-speech tagger for Turkish}
\author{Your Name}
\date{Due date: 16 November 2015 at 11:59 PM Central time}

\begin{document}
\maketitle

\section{Motivation}
\label{sec:motivation}
My main motivation for undertaking this project is to understand the main
principles underlying POS-tagging; due to time and resource constraints, it is
not possible to build a competitive POS-tagger for Turkish without doing
morphological analysis hence this project. Nevertheless, I expect my POS-tagger
to work with a decent accuracy that is good enough to be a starting point for
further development. Moreover, \citet{Korkut2015} began the development of a
Turkish NLP framework implemented in the programming language Haskell. I intend to incorporate the
POS-tagger to the Guguk project, hence contributing towards the existence of a
Turkish NLP-framework in Haskell.

\section{Related Work}

% Briefly survey the most salient prior work that relates to the problem you wish
% to solve. This section should cite relevant sources. This section should be at
% least one to two paragraphs in length.
The most important advancement in the computational linguistics of Turkish is
the NLP framework ``Zemberek'' by \citet{akin2007zemberek}. Although it started
out as an NLP tool for the Turkish language, Zemberek moved on to become a
framework for dealing with all Turkic languages

As we saw is Section \ref{sec:motivation}, ...
%
% For example: 
%

\citet{Weaver_memo_1949} proposed that machine translation \ldots
%
The square root of 4 is 2 \citep{1952_06_17_BarHillel}.


\section{Proposal}

Write your proposal here.


\bibliographystyle{apalike}
\bibliography{your_bibliography}

\end{document}
